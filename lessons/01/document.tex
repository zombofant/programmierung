
\title{Programmierung}
\subtitle{1. Übung}

\newcommand{\s}{\textvisiblespace}
\newcommand{\bs}{\s\s\s\s}

\begin{document}
    \begin{frame}
        \frontframe
    \end{frame}

    \section{Organisatorisches}
    \subsection{~}

    \begin{frame}
        \frametitle{Übungsablauf}
        Primär Präsenzübung. Das heißt:
        \begin{itemize}
        \item Bearbeitung der Aufgabe in der Übungsstunde
        \item Hilfestellungen von mir :)
        \item Abgabe per E-Mail entweder in der Stunde oder von zu Hause nach
            evtl. Schönheitskorrekturen bzw. Dokumentation am Programm
        \end{itemize}
    \end{frame}

    \begin{frame}
        \frametitle{Bewertung}
        \begin{itemize}
        \item 5 Punkte
            \pause
            \begin{itemize}
            \item Funktionalität ca. 75\% der Punkte
            \item 25\% Dokumentation
            \end{itemize}
        \pause
        \item 1 Zusatzpunkt für Form, Eleganz, Kreativität …
        \pause
        \item Abgegebene Übungen: 12 insgesamt, davon eine unbewertet
        \item Zum Bestehen des Moduls erforderlich: 33\,Punkte aus \emph{mindestens
            9 Übungen!}
        \item 100\% bei 55\,Punkten
        \end{itemize}
    \end{frame}

    \begin{frame}
        \frametitle{Reformationstag 31. Oktober}
        \begin{itemize}
        \item Übung fällt aus!
        \pause
        \item Aber: Dafür bearbeitung der Aufgabe selbstständig zu Hause,
            einsenden per E-Mail
        \item Bewertung ebenso wie diese Übung
        \end{itemize}
    \end{frame}

    \section{Python-Syntax addendum}
    \subsection{~}

    \begin{frame}
        %\textvisiblespace
        \frametitle{Whitespace?}
        %xkcd goes here
    \end{frame}

    \begin{frame}
        \frametitle{Einrückung}
        \begin{itemize}
        \item Wo andere Sprachen Klammern, \texttt{begin}/\texttt{end}
            erfordern, erfordert Python Einrückung
        \pause
        \item Vorteil: Code wird automatisch lesbarer
        \item Nachteil: Einige fühlen sich dadurch eingeengt
        \end{itemize}
    \end{frame}

    \begin{frame}
        \frametitle{Einrückung}
        \begin{columns}%
        \column{.45\linewidth}
            \small \ttfamily
            def min(a, b): \\
            \bs if a > b: \\
            \bs \bs return b \\
            \bs else: \\
            \bs \bs return a \\
        \column{.5\linewidth}
            \begin{itemize}
            \item erste vier Leerzeichen markieren den Block für \texttt{def min(a, b)}
            \item zweite vier Leerzeichen markieren den Block für \texttt{if} bzw. \texttt{else}
            \end{itemize}
        \end{columns}
    \end{frame}

    \begin{frame}
        \frametitle{Konvention}
        \begin{itemize}
        \item Python Code Styleguide ist in PEP-0008 festgeschrieben: \\
            \url{http://www.python.org/dev/peps/pep-3333/}
        \item Für diese Übungsreihe gilt:
            \begin{itemize}
            \item Vier Leerzeichen Einrückung (\textbf{keine Tabs})
            \item Korrekte Angabe des Encodings (UTF-8 empfohlen): \\
                \texttt{\# encoding=utf-8}
            \end{itemize}
        \end{itemize}
    \end{frame}

    \section{Aufgabe}\subsection{~}

    \begin{frame}
        \frametitle{Interaktive Python-Shell}
        \begin{enumerate}[1.]
        \item Numerische Operationen
            \begin{enumerate}[a)]
            \item Addition
            \item Subtraktion
            \item Multiplikation
            \item Division und Divisionsrest
            \end{enumerate}
        \item Textoperationen
            \begin{enumerate}[a)]
            \item Verketten
            \item Vervielfachen
            \item Länge bestimmen
            \item Slicing
            \end{enumerate}
        \item Potenzen
        \item Ungerade Zahlen
        \item Primzahlen
        \end{enumerate}
    \end{frame}

\end{document}

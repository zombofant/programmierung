
\title{Programmierung}
\subtitle{2. Übung}

\newcommand{\s}{\textvisiblespace}
\newcommand{\bs}{\s\s\s\s}

\begin{document}
    \begin{frame}
        \frontframe
    \end{frame}

    \section{Vergangene Übung}
    \subsection{~}

    \begin{frame}
        \begin{itemize}
        \item Dokumentation
        \pause
        \item Einrückung (PEP8-ness)
        \pause
        \item \texttt{4/3}
        \end{itemize}
    \end{frame}

    \section{Zukünftige Übung}
    \subsection{~}

    \begin{frame}
        \begin{enumerate}
        \item Bestimmen, ob ein Dreieck rechtwinklig ist anhand der Länge der
            Kanten.
        \item Fakultät von natürlichen Zahlen.
        \item Berechnen der Euler'schen Zahl als Näherung aus der Reihe.
        \item Größter gemeinsamer Teiler.
        \end{enumerate}
    \end{frame}
\end{document}
